\documentclass[12pt]{article}
\usepackage[none]{hyphenat}
\usepackage{fontspec}
\usepackage{geometry}
\usepackage{enumitem}
\usepackage{titlesec}
\usepackage{tikz}
\usepackage{amsmath}
\usepackage{array}
\usepackage{longtable}
\usepackage{xcolor}
\usepackage{listings}
\usepackage{float}
\usetikzlibrary{calc, shapes.geometric, arrows, positioning}

\setmainfont{TH Sarabun New}
\geometry{a4paper, margin=1in}

\lstset{
    basicstyle=\ttfamily\small,
    breaklines=true,
    frame=single,
    rulecolor=\color{black!30},
    backgroundcolor=\color{gray!5},
    language=SQL,
    keywordstyle=\color{blue}
}

\begin{document}
\begin{center}
    \rule{\textwidth}{1pt} \\
    \textbf{\Large Mock Exam: CPE241 Database Systems (Module 1)} \\
    \textbf{Lectures 1-5: Advanced Theory \& Modeling Analysis} \\
    \textbf{\textcolor{red}{HARD MODE}} \\
    \rule{\textwidth}{1pt}
\end{center}

\section*{Part 1: MCQ (English) - 20 Marks}
\begin{enumerate}
    \item In Concurrency Control, a "Wait-for Graph" is used to detect:
    \begin{enumerate}[label=\alph*)] \item Deadlock \quad \item Phantom reads \quad \item Cascading rollbacks \quad \item Dirty writes \end{enumerate}
    \item \dots (Part 1 truncated)
\end{enumerate}

\newpage
\section*{Part 2: Analytical Short Answer (ภาษาไทย) - 20 Marks}
\textbf{คำสั่ง:} จงวิเคราะห์โจทย์และเติมคำตอบลงในตารางให้สมบูรณ์

\begin{longtable}{|c|p{8.5cm}|p{5cm}|}
\hline
\textbf{No.} & \textbf{โจทย์ / สถานการณ์วิเคราะห์} & \textbf{คำตอบ / เหตุผลประกอบ} \\ \hline
1 & จงอธิบายปรากฏการณ์ "Lossless Join" และความสำคัญในการรักษาความถูกต้องของข้อมูล & \\[2.5cm] \hline
2 & "Deadlock" เกิดขึ้นได้อย่างไรในระบบฐานข้อมูล และ DBMS มีวิธีจัดการเบื้องต้นอย่างไร? & \\[2.5cm] \hline
3 & "Natural Join" ต่างจาก "Inner Join" ทั่วไปอย่างไรในเชิงโครงสร้าง? & \\[2.5cm] \hline
4 & ในการออกแบบ ERD, การเลือกระหว่าง "Attribute" และ "Entity" มีเกณฑ์ตัดสินใจอย่างไร? & \\[2.5cm] \hline
5 & จงระบุความแตกต่างระหว่าง "Generalization" และ "Specialization" ใน EER & \\[2.5cm] \hline
\end{longtable}

\newpage
\section*{Part 3: Advanced Practical (ภาษาไทย) - 20 Marks}
\dots (Content truncated)
\end{document}
