\documentclass[a4paper,12pt]{article}

% แพ็กเกจสำหรับภาษาไทยและการจัดการเอกสาร
\usepackage{fontspec}
\usepackage{xunicode}
\usepackage{xltxtra}
\usepackage[margin=2.5cm]{geometry}
\usepackage{enumitem}
\usepackage{listings}
\usepackage[table]{xcolor} % แทนที่ color และรองรับการสลับสีตาราง
\usepackage{graphicx}
\usepackage{longtable} % สำหรับตารางที่ขึ้นหน้าใหม่ได้
\usepackage{booktabs} % สำหรับเส้นแบ่งตารางที่สวยงาม
\usepackage{tikz} % สำหรับวาด ER Diagram
\usetikzlibrary{shapes.geometric, arrows, positioning, fit, calc}
\usepackage{float}
\usepackage[normalem]{ulem}
\usepackage{needspace} % ป้องกันการตัดหน้าในข้อสอบ

% ตั้งค่าฟอนต์ภาษาไทย (ต้องมีฟอนต์ TH Sarabun New)
\setmainfont{TH Sarabun New} 
\setmonofont{Courier New} % สำหรับ code ให้มองเห็นง่ายขึ้น

% ปรับแต่งค่าความห่างบรรทัดให้พอดีกับภาษาไทย
\linespread{1.15}

% ตั้งค่าสีสำหรับ Code SQL
\definecolor{codegreen}{rgb}{0,0.6,0}
\definecolor{codegray}{rgb}{0.5,0.5,0.5}
\definecolor{codepurple}{rgb}{0.58,0,0.82}
\definecolor{backcolour}{rgb}{0.95,0.95,0.92}

\lstdefinestyle{mystyle}{
    backgroundcolor=\color{backcolour},   
    commentstyle=\color{codegreen},
    keywordstyle=\color{blue},
    numberstyle=\tiny\color{codegray},
    stringstyle=\color{codepurple},
    basicstyle=\ttfamily\footnotesize,
    breakatwhitespace=false,         
    breaklines=true,                 
    captionpos=b,                    
    keepspaces=true,                 
    numbers=left,                    
    numbersep=5pt,                  
    showspaces=false,                
    showstringspaces=false,
    showtabs=false,                  
    tabsize=2
}
\lstset{style=mystyle, language=SQL}

% TikZ Styles สำหรับวาด ERD
\tikzstyle{entity} = [rectangle, draw, text centered, minimum height=2em, minimum width=3em]
\tikzstyle{weakentity} = [rectangle, draw, double, double distance=2pt, text centered, minimum height=2em, minimum width=3em]
\tikzstyle{relationship} = [diamond, draw, text centered, aspect=2]
\tikzstyle{identrelationship} = [diamond, draw, double, double distance=2pt, text centered, aspect=2]
\tikzstyle{attribute} = [ellipse, draw, text centered, node distance=2cm]
\tikzstyle{keyattribute} = [ellipse, draw, text centered, node distance=2cm, font=\bfseries] % Underline manual in node text
\tikzstyle{multiattribute} = [ellipse, draw, double, double distance=2pt, text centered]
\tikzstyle{line} = [draw, -]

\begin{document}

\begin{center}
    \textbf{\Large CPE 241 Database Systems (Module 1)}\\
    \textbf{Mock Exam (ข้อสอบจำลอง)}\\
    \vspace{0.2cm}
    วันเวลาสอบ: วันศุกร์ที่ 13 กุมภาพันธ์ 2569 | เวลา: 09:00 - 11:00 (2 ชั่วโมง)\\
    คะแนนเต็ม 60 คะแนน
\end{center}
\hrule
\vspace{0.5cm}

\section*{Part 1: Multiple Choice Questions (20 Marks)}
\textbf{Instructions:} Choose the most appropriate answer. (1 Mark each)

\begin{enumerate}[label=\textbf{\arabic*.}]
    \needspace{6\baselineskip}
    % Target: B (Edgar F. Codd)
    \item Who is considered the father of the Relational Database Model?
    \begin{enumerate}[label=\alph*)]
        \item Peter Chen
        \item Edgar F. Codd
        \item Gordon Everest
        \item Charles Bachman
    \end{enumerate}

    \needspace{6\baselineskip}
    % Target: D (Duplicate rows)
    \item Which of the following is \textbf{NOT} a characteristic of a Relation?
    \begin{enumerate}[label=\alph*)]
        \item Order of rows is insignificant.
        \item Order of columns is insignificant.
        \item Each cell contains atomic values.
        \item Duplicate rows are allowed.
    \end{enumerate}

    \needspace{6\baselineskip}
    % Target: C (Composite)
    \item An attribute representing "Name" which consists of "Firstname" and "Lastname" is classified as:
    \begin{enumerate}[label=\alph*)]
        \item Single-valued Attribute
        \item Derived Attribute
        \item Composite Attribute
        \item Multi-valued Attribute
    \end{enumerate}

    \needspace{6\baselineskip}
    % Target: C (3 Tables)
    \item When mapping a binary \textbf{Many-to-Many (M:N)} relationship into a relational schema, how many tables (relations) are created in total?
    \begin{enumerate}[label=\alph*)]
        \item 1
        \item 2
        \item 3
        \item 4
    \end{enumerate}

    \needspace{6\baselineskip}
    % Target: A (1:1)
    \item Which type of relationship can be collapsed (merged) into a single table to optimize space?
    \begin{enumerate}[label=\alph*)]
        \item One-to-One (1:1)
        \item One-to-Many (1:N)
        \item Many-to-Many (M:N)
        \item Recursive M:N
    \end{enumerate}

    \needspace{6\baselineskip}
    % Target: D (WHERE)
    \item Which SQL clause is required to delete specific rows in a \texttt{DELETE} statement?
    \begin{enumerate}[label=\alph*)]
        \item FROM
        \item GROUP BY
        \item ORDER BY
        \item WHERE
    \end{enumerate}

    \needspace{6\baselineskip}
    % Target: B (Entity Integrity)
    \item What does the \textbf{Entity Integrity Constraint} state?
    \begin{enumerate}[label=\alph*)]
        \item Foreign Key cannot be NULL.
        \item Primary Key cannot be NULL.
        \item Attributes must match their domains.
        \item Relationships must be consistent.
    \end{enumerate}

    \needspace{6\baselineskip}
    % Target: B (Weak Entity)
    \item In Chen Notation, what does a \textbf{double rectangle} represent?
    \begin{enumerate}[label=\alph*)]
        \item Strong Entity
        \item Weak Entity
        \item Associative Entity
        \item Composite Attribute
    \end{enumerate}

    \needspace{6\baselineskip}
    % Target: A (INSERT)
    \item Which of the following commands belongs to \textbf{DML} (Data Manipulation Language)?
    \begin{enumerate}[label=\alph*)]
        \item INSERT
        \item CREATE
        \item ALTER
        \item DROP
    \end{enumerate}

    \needspace{6\baselineskip}
    % Target: D (M:1)
    \item If an Employee belongs to only 1 Department, but a Department has many Employees, what is the relationship type from Employee to Department?
    \begin{enumerate}[label=\alph*)]
        \item One-to-One
        \item One-to-Many
        \item Many-to-Many
        \item Many-to-One
    \end{enumerate}

    \needspace{6\baselineskip}
    % Target: C (Derived)
    \item An attribute that is not physically stored but calculated from other attributes (e.g., Age from Birthdate) is called:
    \begin{enumerate}[label=\alph*)]
        \item Key Attribute
        \item Stored Attribute
        \item Derived Attribute
        \item Multi-valued Attribute
    \end{enumerate}

    \needspace{6\baselineskip}
    % Target: A (Disjoint)
    \item In EER Diagrams, what does the circle with a "d" symbol represent?
    \begin{enumerate}[label=\alph*)]
        \item Disjoint Constraint
        \item Overlapping Constraint
        \item Total Participation
        \item Partial Participation
    \end{enumerate}

    \needspace{6\baselineskip}
    % Target: D (DROP)
    \item Which SQL command removes a table \textbf{completely} (both structure and data) from the database?
    \begin{enumerate}[label=\alph*)]
        \item DELETE TABLE
        \item TRUNCATE TABLE
        \item REMOVE TABLE
        \item DROP TABLE
    \end{enumerate}

    \needspace{6\baselineskip}
    % Target: C (Weak ID)
    \item How is a Weak Entity identified?
    \begin{enumerate}[label=\alph*)]
        \item By its own Primary Key.
        \item By a surrogate key.
        \item By its Partial Key and its Owner's Primary Key.
        \item It does not need identification.
    \end{enumerate}

    \needspace{6\baselineskip}
    % Target: A (Recursive)
    \item How do you map a \textbf{Recursive 1:N} relationship (e.g., Employee manages Employees) to a table?
    \begin{enumerate}[label=\alph*)]
        \item Add a Foreign Key referencing the same table's Primary Key.
        \item Create a new relationship table.
        \item Use a composite primary key.
        \item It is impossible in the Relational Model.
    \end{enumerate}

    \needspace{6\baselineskip}
    % Target: C (Domain)
    \item Which constraint ensures that values in a column must match a defined data type or range?
    \begin{enumerate}[label=\alph*)]
        \item Key Constraint
        \item Referential Integrity Constraint
        \item Domain Constraint
        \item Entity Integrity Constraint
    \end{enumerate}

    \needspace{6\baselineskip}
    % Target: A (Candidate)
    \item A minimal super key that uniquely identifies a tuple is called:
    \begin{enumerate}[label=\alph*)]
        \item Candidate Key
        \item Foreign Key
        \item Composite Key
        \item Alternate Key
    \end{enumerate}

    \needspace{6\baselineskip}
    % Target: D (Junction)
    \item If "Order" has many "Products" and "Product" is in many "Orders", how is this mapped?
    \begin{enumerate}[label=\alph*)]
        \item Merge into Order table.
        \item Put Product FK in Order table.
        \item Put Order FK in Product table.
        \item Create a new Junction Table (e.g., Order\_Items).
    \end{enumerate}

    \needspace{6\baselineskip}
    % Target: B (Total)
    \item What does a \textbf{double line} connecting an entity to a relationship represent?
    \begin{enumerate}[label=\alph*)]
        \item Partial Participation
        \item Total Participation
        \item Cardinality of 'Many'
        \item Weak Relationship
    \end{enumerate}

    \needspace{6\baselineskip}
    % Target: A (DDL)
    \item The command \texttt{ALTER TABLE} falls under which category?
    \begin{enumerate}[label=\alph*)]
        \item DDL
        \item DML
        \item DCL
        \item TCL
    \end{enumerate}
\end{enumerate}

\newpage

\section*{ตอนที่ 2: อัตนัยแบบเขียนตอบสั้น (Short Answer) (20 คะแนน)}
\textbf{คำสั่ง:} จงตอบคำถามต่อไปนี้ให้ถูกต้อง (ข้อละ 2 คะแนน)

\begin{enumerate}[label=\textbf{\arabic*.}]
    \needspace{3\baselineskip}
    \item หากต้องการ \textbf{แก้ไขโครงสร้างตาราง} (เช่น เปลี่ยนชื่อคอลัมน์ หรือเปลี่ยน Data Type) ต้องใช้คำสั่ง SQL ใด?
    \vspace{0.8cm}
    
    \needspace{3\baselineskip}
    \item จงบอกชื่อ \textbf{rDBMS} ที่นิสิตใช้งานจริงในการทำ Lab ของวิชานี้ (Module 1)
    \vspace{0.8cm}
    
    \needspace{3\baselineskip}
    \item คำสั่งกลุ่ม \textbf{DML} (Data Manipulation Language) ประกอบด้วยคำสั่งอะไรบ้าง? (ตอบให้ครบ 3 คำสั่งหลัก)
    \vspace{0.8cm}
    
    \needspace{6\baselineskip}
    \item จงระบุประเภทความสัมพันธ์ (Cardinality) ของคู่ Entity ต่อไปนี้ (เช่น 1:1, 1:N, M:N):
    \begin{itemize}
        \item \textbf{Mother - Child} : \underline{\hspace{3cm}}
        \item \textbf{Doctor - Patient} : \underline{\hspace{3cm}}
        \item \textbf{House - Address} : \underline{\hspace{3cm}}
    \end{itemize}
    \vspace{0.5cm}

    \needspace{3\baselineskip}
    \item จากข้อ 4 คู่ความสัมพันธ์ Entity ใดบ้างที่เป็น \textbf{Weak Relationship}?
    \vspace{0.8cm}
    
    \needspace{3\baselineskip}
    \item \textbf{Composite Key} คืออะไร? จงอธิบายสั้นๆ
    \vspace{0.8cm}
    
    \needspace{3\baselineskip}
    \item ใน EER Diagram สัญลักษณ์ \textbf{"o"} (Overlapping) มีความหมายว่าอย่างไร?
    \vspace{0.8cm}
    
    \needspace{3\baselineskip}
    \item หากต้องการลบข้อมูลทั้งหมดในตารางอย่างรวดเร็ว โดยยังเก็บโครงสร้างตารางไว้ และ Reset ค่า Auto-increment ควรใช้คำสั่งใด?
    \vspace{0.8cm}
    
    \needspace{3\baselineskip}
    \item \textbf{Partial Key} คืออะไร และใช้สัญลักษณ์ใดใน Chen Notation?
    \vspace{0.8cm}
    
    \needspace{3\baselineskip}
    \item ในขั้นตอน Mapping หากพบ \textbf{Multi-valued Attribute} (เช่น พนักงาน 1 คน มีหลายเบอร์โทร) ต้องดำเนินการอย่างไร?
    \vspace{0.8cm}
\end{enumerate}

\newpage

\section*{ตอนที่ 3: อัตนัยแบบแสดงวิธีทำ (Long Answer) (20 คะแนน)}
\textbf{คำสั่ง:} จงแสดงวิธีทำตามโจทย์ที่กำหนดให้

\begin{enumerate}[label=\textbf{ข้อที่ \arabic*.}]
    \needspace{10\baselineskip}
    \item \textbf{จงเขียนคำสั่งทั้งหมดในการสร้างตารางพร้อมเพิ่มตัวอย่างข้อมูลทั้งสามแถวให้เหมือนภาพข้างล่าง โดยกำหนดให้}
    \begin{enumerate}[label=\alph*.]
        \item \texttt{student\_id} เป็น \textbf{Primary Key}
        \item \texttt{firstname} และ \texttt{lastname} ต้อง \textbf{ไม่เป็น NULL}
    \end{enumerate}
    
    \begin{center}
    \renewcommand{\arraystretch}{1.5}
    \begin{tabular}{|c|c|c|c|c|}
    \hline
    \rowcolor[gray]{0.9} \textbf{student\_id} & \textbf{firstname} & \textbf{lastname} & \textbf{age} & \textbf{tel} \\ \hline
    27001 & Luffy & Monkey & 18 & 1234567890 \\ \hline
    27002 & Chopper & Tony & NULL & 0123456789 \\ \hline
    27003 & Roronoa & Zoro & 17 & NULL \\ \hline
    \end{tabular}
    \end{center}

    \vspace{6cm} % เว้นที่ว่างสำหรับเขียนตอบ (ไม่มีกรอบ)

    \needspace{15\baselineskip}
    \item \textbf{จงสร้าง Conceptual level design โดยใช้ Chen notation ER diagram จากข้อมูลต่อไปนี้}
    \begin{enumerate}[label=\alph*.]
        \item บริษัทมีหลายแผนกซึ่งเก็บรหัสและชื่อแผนกไว้โดยชื่อไม่ซ้ำกัน
        \item แต่ละแผนกมีพนักงานโดยเก็บข้อมูลรหัสพนักงาน ชื่อ นามสกุล ที่อยู่ อายุซึ่งพนักงานแต่ละคนสังกัดได้เพียงแผนกเดียวเท่านั้น
        \item ในกรณีที่พนักงานมีลูก (Child) ให้เก็บชื่อลูกและวันเกิด
    \end{enumerate}

    \vspace{10cm} % เว้นที่ว่างสำหรับวาดรูป (ไม่มีกรอบ)

\end{enumerate}

\newpage

\section*{เฉลยละเอียด (Detailed Answer Key)}

\subsection*{Part 1: Multiple Choice Explanations}

\renewcommand{\arraystretch}{2.0}
\rowcolors{2}{gray!10}{white}
\begin{longtable}{|c|c|p{13.5cm}|}
\hline
\rowcolor[gray]{0.8} \textbf{No.} & \textbf{Ans} & \textbf{คำอธิบายอย่างละเอียด (Detailed Explanation)} \\ \hline
1 & \textbf{b} & \textbf{Edgar F. Codd} (IBM) เป็นผู้คิดค้น Relational Model ในปี 1970 ส่วน Peter Chen คิดค้น ER Diagram (1976) และ Charles Bachman คิดค้น Network Model \\ \hline
2 & \textbf{d} & ตามคุณสมบัติของ Relation ข้อมูลแต่ละแถว (Tuple) จะต้องไม่ซ้ำกัน (\textbf{No duplicate rows}) เพื่อให้ใช้ Primary Key ระบุตัวตนได้ ส่วนลำดับของแถวและคอลัมน์นั้นไม่มีความสำคัญ \\ \hline
3 & \textbf{c} & \textbf{Composite Attribute} คือ Attribute ที่สามารถแตกย่อยออกเป็นส่วนๆ ได้ (เช่น Name แตกเป็น Firstname, Lastname) ส่วน Multivalued คือมีหลายค่าในตัวเดียว (เช่น เบอร์โทรศัพท์หลายเบอร์) \\ \hline
4 & \textbf{c} & การแปลง M:N จะต้องสร้างตารางใหม่ขึ้นมาเป็น \textbf{Junction Table} เพื่อเก็บความสัมพันธ์ รวมกับตาราง Entity เดิม 2 ตาราง จึงรวมเป็น 3 ตารางเสมอ \\ \hline
5 & \textbf{a} & ความสัมพันธ์แบบ \textbf{1:1} ที่มี Total Participation ทั้งสองฝั่ง สามารถยุบรวมเป็นตารางเดียวได้เพื่อลด Overhead ในการ Join และประหยัดเนื้อที่ \\ \hline
6 & \textbf{d} & คำสั่ง \textbf{DELETE} หากไม่มี WHERE clause จะส่งผลให้ระบบไล่ลบข้อมูลทุกแถวที่อยู่ในตารางจนหมด แต่โครงสร้างตาราง (Schema) ยังคงอยู่ \\ \hline
7 & \textbf{b} & \textbf{Entity Integrity} กำหนดว่า Primary Key ห้ามเป็น NULL และต้อง Unique ส่วน Referential Integrity เกี่ยวข้องกับการใช้ Foreign Key อ้างอิงไปยังตารางอื่น \\ \hline
8 & \textbf{b} & ในสัญลักษณ์ของ Chen Notation \textbf{สี่เหลี่ยมขอบคู่ (Double Rectangle)} หมายถึง Weak Entity ที่ไม่สามารถระบุตัวตนได้ด้วยตัวเอง ต้องพึ่งพา Owner Entity \\ \hline
9 & \textbf{a} & \textbf{DML} ใช้จัดการข้อมูล ได้แก่ INSERT (เพิ่ม), UPDATE (แก้ไข), DELETE (ลบ) ส่วน DDL ใช้จัดการโครงสร้าง เช่น CREATE, ALTER, DROP \\ \hline
10 & \textbf{d} & พนักงาน 1 คนสังกัด 1 แผนก แต่แผนกมีได้หลายคน เมื่อมองจากมุมมองพนักงานไปแผนกจึงเป็น \textbf{Many-to-One} (N:1) \\ \hline
11 & \textbf{c} & \textbf{Derived Attribute} คือค่าที่ไม่ได้เก็บจริงในฐานข้อมูลแต่คำนวณเอาจากค่าอื่น (เช่น Age คำนวณจาก Birthdate) ใช้สัญลักษณ์วงรีเส้นประ (Dashed Ellipse) \\ \hline
12 & \textbf{a} & \textbf{Disjoint (d)} หมายถึงสมาชิกของ Superclass สามารถเป็นสมาชิกของ Subclass ได้เพียงกลุ่มเดียวเท่านั้น (เลือกได้อย่างใดอย่างหนึ่ง) ห้ามทับซ้อนกัน \\ \hline
13 & \textbf{d} & \textbf{DROP TABLE} จะลบทั้งข้อมูลและโครงสร้างตารางออกจากสารบบของฐานข้อมูลทันที ไม่สามารถกู้คืนได้ง่ายๆ \\ \hline
14 & \textbf{c} & \textbf{Weak Entity} ต้องใช้ Partial Key (Discriminator) ของตัวเอง รวมกับ Primary Key ของ Owner Entity ถึงจะระบุตัวตน (Identify) ได้ \\ \hline
15 & \textbf{a} & \textbf{Recursive Relationship} คือความสัมพันธ์ภายในตารางเดียวกัน วิธีแก้คือเพิ่ม Foreign Key 1 คอลัมน์ที่อ้างอิงกลับมายัง Primary Key ของตารางตัวเอง \\ \hline
16 & \textbf{c} & \textbf{Domain Constraint} คือข้อกำหนดเกี่ยวกับช่วงข้อมูลหรือชนิดข้อมูลที่อนุญาตให้ใส่ในคอลัมน์นั้นๆ (เช่น คะแนนต้องอยู่ระหว่าง 0-100) \\ \hline
17 & \textbf{a} & \textbf{Candidate Key} คือ Super Key ที่มีจำนวนคุณลักษณะน้อยที่สุด (Minimal) ซึ่งมีคุณสมบัติเพียงพอจะเป็น Primary Key ได้ \\ \hline
18 & \textbf{d} & เนื่องจาก Order และ Product สัมพันธ์กันแบบ M:N (สินค้า 1 อย่างอยู่ในหลายออเดอร์) จึงต้องสร้าง \textbf{Junction Table} (เช่น Order\_Items) ขึ้นมาคั่นกลางเสมอ \\ \hline
19 & \textbf{b} & \textbf{เส้นคู่ (Double Line)} หมายถึงทุกๆ Entity ในกลุ่มนั้น "ต้อง" มีความสัมพันธ์กับอีกฝั่งหนึ่งเสมอ เรียกว่า Total Participation \\ \hline
20 & \textbf{a} & \textbf{ALTER TABLE} เป็นคำสั่งแก้ไขนิยามข้อมูล (Data Definition) เช่น การเพิ่มคอลัมน์ หรือเปลี่ยนชนิดข้อมูล จึงจัดเป็น DDL \\ \hline
\end{longtable}

\subsection*{Part 2: Short Answer Solutions}

\renewcommand{\arraystretch}{2.2}
\rowcolors{2}{gray!10}{white}
\begin{longtable}{|c|p{5.2cm}|p{9.8cm}|}
    \hline
    \rowcolor[gray]{0.8} \textbf{No.} & \textbf{คำตอบ (Solutions)} & \textbf{คำอธิบายเพิ่มเติม (Detailed Explanation)} \\ \hline
    1 & \texttt{ALTER TABLE} & ใช้ในการแก้ไข Schema เช่น เพิ่ม/ลดคอลัมน์ หรือเปลี่ยน Data Type ของคอลัมน์ที่มีอยู่แล้ว \\ \hline
    2 & \textbf{MySQL} หรือ \textbf{MariaDB} & เป็นระบบจัดการฐานข้อมูล (rDBMS) แบบ Open-source ที่นิยมใช้คู่กับ phpMyAdmin ในการทำ Lab \\ \hline
    3 & \texttt{INSERT}, \texttt{UPDATE}, \texttt{DELETE} & ชุดคำสั่งพื้นฐานในการ "จัดการข้อมูล" (Manipulation) ภายในตารางฐานข้อมูล \\ \hline
    4 & \textbf{1:N, M:N, 1:1} & แม่ 1 คนมีลูกได้หลายคน (1:N), หมอ 1 คนรักษาคนไข้ได้หลายคนและคนไข้ 1 คนหาหมอได้หลายท่าน (M:N), บ้าน 1 หลังมีเลขที่บ้านได้เพียง 1 เดียว (1:1) \\ \hline
    5 & \textbf{Child} (หรือลูก) & เนื่องจากลูกไม่มีตัวตนที่สมบูรณ์หากไม่มีแม่ (Mother) จึงเป็นเอนทิตีที่ต้องพึ่งพา (Dependent) เอนทิตีหลักเสมอ \\ \hline
    6 & \textbf{Composite Key} & การใช้ Attribute มากกว่า 1 ตัวรวมกันทำหน้าที่เป็น Primary Key เพื่อให้แน่ใจว่าแต่ละแถวมีความ Unique จริงๆ \\ \hline
    7 & \textbf{Overlapping} (o) & เป็นข้อกำหนดที่ยอมรับให้ Entity ตัวหนึ่งสามารถปรากฏอยู่ใน Subclass หลายๆ กลุ่มพร้อมกันได้ (เช่น เป็นทั้งพนักงานและนักศึกษา) \\ \hline
    8 & \texttt{TRUNCATE TABLE} & คำสั่งที่ใช้ลบข้อมูลทั้งหมดทิ้งทันที โดยผลลัพธ์คือตารางว่างเปล่าและ Reset ค่าเลขรันอัตโนมัติ (Identity/Auto-increment) ใหม่ \\ \hline
    9 & \textbf{Partial Key} & คือ Attribute ที่ใช้แยกแยะความแตกต่างของ Weak Entity ภายใต้ Owner เดียวกัน สัญลักษณ์คือวงรีขีดเส้นใต้ประ (\textbf{Dashed Underline}) \\ \hline
    10 & \textbf{สร้างตารางใหม่ (New Table)} & ค่าที่มีหลายค่าต้องถูกแยกออกมาสร้างตารางใหม่ โดยดึง Primary Key จากตารางเดิมมาเป็น Foreign Key ร่วมกับค่านั้นๆ \\ \hline
\end{longtable}

\newpage
\subsection*{Part 3: Long Answer Solutions (with Visuals)}

\textbf{ข้อ 1 (SQL Solution):}
\begin{lstlisting}
CREATE TABLE students (
    student_id INT PRIMARY KEY,
    firstname VARCHAR(50) NOT NULL,
    lastname VARCHAR(50) NOT NULL,
    age INT,
    tel VARCHAR(20)
);

INSERT INTO students VALUES (27001, 'Luffy', 'Monkey', 18, '1234567890');
INSERT INTO students VALUES (27002, 'Chopper', 'Tony', NULL, '0123456789');
INSERT INTO students VALUES (27003, 'Roronoa', 'Zoro', 17, NULL);
\end{lstlisting}

\textbf{ข้อ 2 (ER Diagram Solution):}
\begin{center}
\begin{tikzpicture}[node distance=2.5cm, every edge/.style={draw, thick}, scale=0.8, transform shape]
    % Entities
    \node[entity] (dept) {Department};
    \node[entity] (emp) [right=5cm of dept] {Employee};
    \node[weakentity] (dep) [below=3cm of emp] {Child};

    % Attributes for Department
    \node[keyattribute] (dnum) [above left=0.5cm and 0.5cm of dept] {\underline{Dept\_ID}} edge (dept);
    \node[attribute] (dname) [below left=0.5cm and 0.5cm of dept] {Name} edge (dept);

    % Attributes for Employee
    \node[keyattribute] (eid) [above=1cm of emp] {\underline{Emp\_ID}} edge (emp);
    \node[attribute] (ename) [above right=0.5cm and 0.5cm of emp] {FName} edge (emp);
    \node[attribute] (esur) [right=1cm of emp] {LName} edge (emp);
    \node[attribute] (eaddr) [below right=0.5cm and 0.5cm of emp] {Address} edge (emp);

    % Attributes for Child
    \node[attribute] (depname) [left=1.5cm of dep] {\dashuline{Name}} edge (dep); % Partial Key
    \node[attribute] (depbirth) [right=1.5cm of dep] {Birthdate} edge (dep);

    % Relationships (Manually positioned at center of paths)
    \node[relationship] (belong) at ($(dept)!0.5!(emp)$) {Works\_in};
    \node[identrelationship] (has) at ($(emp)!0.5!(dep)$) {Has};

    % Connectors with Cardinality
    \draw (dept) -- (belong) node[pos=0.3, above=0.1cm] {1};
    \draw (belong) -- (emp) node[pos=0.7, above=0.1cm] {N};
    
    \draw (emp) -- (has) node[pos=0.3, left=0.1cm] {1};
    \draw[double, thick] (has) -- (dep) node[pos=0.7, left=0.1cm] {N}; 
\end{tikzpicture}
\end{center}

\textbf{คำอธิบายจุดสำคัญ (Detailed Point Analysis):}
\begin{itemize}
    \item \textbf{Department - Employee:} เป็นความสัมพันธ์แบบ 1:N เพราะโจทย์ระบุว่า "พนักงานแต่ละคนสังกัดได้เพียงแผนกเดียว" (ฝั่งพนักงานเป็น Many ฝั่งแผนกเป็น One) โดยใช้เส้นเดี่ยวเชื่อมต่อปกติ (Partial Participation)
    \item \textbf{Child (ลูก):} ถูกออกแบบเป็น \textbf{Weak Entity} (สี่เหลี่ยมขอบคู่) เนื่องจากข้อมูลลูกจะหายไปทันทีหากไม่มีพนักงานคนนั้นอยู่ในระบบ 
    \item \textbf{Name (ชื่อลูก):} ทำหน้าที่เป็น \textbf{Partial Key} (วงรีขีดเส้นใต้ประ) เพราะลูกหลายคนอาจมีชื่อซ้ำกันได้ ต้องระบุตัวตนร่วมกับรหัสพนักงาน (Emp\_ID)
    \item \textbf{Relationship (Has):} เป็น \textbf{Identifying Relationship} (ข้าวหลามตัดขอบคู่) ทำงานร่วมกับความสัมพันธ์ของเอนทิตีอ่อนแอ
    \item \textbf{Participation (ฝั่งลูก):} เป็น \textbf{Total Participation} (เส้นคู่) เนื่องจากลูกทุกคนที่ถูกเก็บข้อมูล "จำเป็น" ต้องสังกัดอยู่กับพนักงานคนใดคนหนึ่งเสมอ
\end{itemize}

\end{document}
