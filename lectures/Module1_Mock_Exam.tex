\documentclass[a4paper,11pt]{article}

\usepackage{fontspec}
\usepackage{xunicode}
\usepackage{xltxtra}
\usepackage[margin=2.5cm]{geometry}
\usepackage{enumitem}
\usepackage{listings}
\usepackage{color}
\usepackage{graphicx}
\usepackage{tikz}
\usepackage{float}
\usepackage[normalem]{ulem}
\usepackage{longtable}
\usepackage{array}
\usetikzlibrary{calc, shapes.geometric, arrows, positioning}

% ตั้งค่าฟอนต์ภาษาไทย
\setmainfont{TH Sarabun New} 

% TikZ Styles สำหรับเฉลย
\tikzset{
    entity/.style={rectangle, draw, minimum width=2.5cm, minimum height=1cm, text centered, thick},
    weakEntity/.style={rectangle, draw, double, minimum width=2.5cm, minimum height=1cm, text centered, thick},
    relationship/.style={diamond, draw, aspect=2, minimum width=2.5cm, text centered, thick},
    identRelationship/.style={diamond, draw, double, aspect=2, minimum width=2.5cm, text centered, thick},
    attribute/.style={ellipse, draw, minimum width=1.5cm, text centered, thin},
    keyAttribute/.style={ellipse, draw, thick, text centered},
    partialKey/.style={ellipse, draw, dashed, text centered, thin}
}

% ตั้งค่าสีสำหรับ Code SQL
\definecolor{codegreen}{rgb}{0,0.6,0}
\definecolor{codegray}{rgb}{0.5,0.5,0.5}
\definecolor{codepurple}{rgb}{0.58,0,0.82}
\definecolor{backcolour}{rgb}{0.95,0.95,0.92}

\lstset{
    backgroundcolor=\color{backcolour},   
    commentstyle=\color{codegreen},
    keywordstyle=\color{blue},
    numberstyle=\tiny\color{codegray},
    stringstyle=\color{codepurple},
    basicstyle=\ttfamily\footnotesize,
    breaklines=true,                 
    frame=single,
    language=SQL
}

\title{\textbf{Mock Exam: CPE241 Database Systems (Module 1)}\\ \large ฉบับมาตรฐาน (Standard Edition) - อ้างอิงแนวรุ่นพี่}
\author{Lectures 1-5 | English MCQs | Thai Written Parts}
\date{คะแนนเต็ม 60 คะแนน | เวลาสอบ 2 ชั่วโมง}

\begin{document}

\maketitle

\section*{Part 1: 20 Multiple Choice Questions (English)}
\textbf{Instructions:} Select the most appropriate answer for each question. (1 mark each)

\begin{enumerate}[label=\textbf{\arabic*.}]
    \item Which of the following is \textbf{NOT} a characteristic of a Relation in a Relational Database?
    \begin{enumerate}[label=\alph*)]
        \item Every row must be identifiable (Unique)
        \item Attributes must have a defined Domain
        \item It allows duplicated rows
        \item The order of rows is insignificant
    \end{enumerate}

    \item Who originally proposed the Relational Database Model in 1970?
    \begin{enumerate}[label=\alph*)]
        \item Peter Chen \quad \item Edgar F. Codd \quad \item Charles Bachman \quad \item Gordon Everest
    \end{enumerate}

    \item If "firstname" and "lastname" are stored separately, they are classified as which type of attribute?
    \begin{enumerate}[label=\alph*)]
        \item Unique attribute \quad \item Composite attribute \quad \item Multi-valued attribute \quad \item Derived attribute
    \end{enumerate}

    \item When mapping a Binary Many-to-Many (M:N) relationship into tables, how many tables are produced in total?
    \begin{enumerate}[label=\alph*)]
        \item 1 \quad \item 2 \quad \item 3 \quad \item 4
    \end{enumerate}

    \item Which of the following relationships can be most effectively "collapsed" into a single table?
    \begin{enumerate}[label=\alph*)]
        \item One-to-one \quad \item One-to-many \quad \item Many-to-many \quad \item Self-reference
    \end{enumerate}

    \item To mapped tables in a specific Binary Relationship and reduce the total number of tables, which relationship type is required?
    \begin{enumerate}[label=\alph*)]
        \item One-to-one \quad \item One-to-many \quad \item Many-to-many \quad \item Multi-valued
    \end{enumerate}

    \item Which SQL clause is used to handle specific rows in a \texttt{DELETE} statement?
    \begin{enumerate}[label=\alph*)]
        \item HAVING \quad \item GROUP BY \quad \item ORDER BY \quad \item WHERE
    \end{enumerate}

    \item A Double Rectangle symbol in Chen Notation ERD represents:
    \begin{enumerate}[label=\alph*)]
        \item Strong Entity \quad \item Weak Entity \quad \item Multi-valued Attribute \quad \item Identifying Relationship
    \end{enumerate}

    \item Which SQL command is used to add a new column to an existing table?
    \begin{enumerate}[label=\alph*)]
        \item UPDATE \quad \item MODIFY \quad \item ALTER \quad \item INSERT
    \end{enumerate}

    \item In the context of the 3-Schema Architecture, physical file organization occurs at which level?
    \begin{enumerate}[label=\alph*)]
        \item External \quad \item Conceptual \quad \item Internal \quad \item Logical
    \end{enumerate}

    \item Attribute which cannot be further subdivided is called:
    \begin{enumerate}[label=\alph*)]
        \item Composite \quad \item Atomic \quad \item Multi-valued \quad \item Derived
    \end{enumerate}

    \item Which notation represents "Total Participation" (Mandatory) in ER Modeling?
    \begin{enumerate}[label=\alph*)]
        \item Single Line \quad \item Dashed Line \quad \item Double Line \quad \item Arrowhead
    \end{enumerate}

    \item The attribute used to identify a Weak Entity in conjunction with the Owner's key is:
    \begin{enumerate}[label=\alph*)]
        \item Primary Key \quad \item Foreign Key \quad \item Partial Key (Discriminator) \quad \item Unique Key
    \end{enumerate}

    \item Which SQL group is used for modifying the database structure?
    \begin{enumerate}[label=\alph*)]
        \item DML \quad \item DDL \quad \item DCL \quad \item TCL
    \end{enumerate}

    \item In a 1:N relationship, where is the Foreign Key (FK) typically placed?
    \begin{enumerate}[label=\alph*)]
        \item On the "one" side \quad \item On the "many" side \quad \item In a junction table \quad \item On both sides
    \end{enumerate}

    \item A Primary Key must satisfy which of the following?
    \begin{enumerate}[label=\alph*)]
        \item Must be unique \quad \item Must not be null \quad \item Both a and b \quad \item None of the above
    \end{enumerate}

    \item Which SQL statement is used to remove all data from a table without deleting the structure?
    \begin{enumerate}[label=\alph*)]
        \item DROP \quad \item DELETE \quad \item REMOVE \quad \item ERASE
    \end{enumerate}

    \item An ER relationship where an Entity participates with itself is called:
    \begin{enumerate}[label=\alph*)]
        \item Binary \quad \item Ternary \quad \item Recursive \quad \item Weak
    \end{enumerate}

    \item Which constraint ensures that a Foreign Key value must match an existing Primary Key?
    \begin{enumerate}[label=\alph*)]
        \item Entity Integrity \quad \item Referential Integrity \quad \item Domain Integrity \quad \item Key Integrity
    \end{enumerate}

    \item In SQL, the command to permanently save changes made in a transaction is:
    \begin{enumerate}[label=\alph*)]
        \item SAVE \quad \item COMMIT \quad \item ROLLBACK \quad \item GRANT
    \end{enumerate}
\end{enumerate}

\newpage

\section*{Part 2: 10 short-answers questions (ภาษาไทย)}
\textbf{คำสั่ง:} จงตอบคำถามต่อไปนี้ให้ถูกต้อง (ข้อละ 2 คะแนน)

\begin{itemize}[itemsep=0.8cm]
    \item ในคำสั่ง SQL หากต้องการ \textbf{เปลี่ยนแปลงโครงสร้าง} ข้อมูลต้องใช้คำสั่งใด? \\
    \textbf{ตอบ:} \rule{10cm}{0.4pt}
    
    \item \textbf{rDBMS} ที่นิสิตใช้งานจริงใน Lab ชื่อว่าอะไร? \\
    \textbf{ตอบ:} \rule{10cm}{0.4pt}
    
    \item กลุ่มคำสั่ง SQL ที่ใช้ในการ \textbf{เพิ่ม ลบ เปลี่ยนแปลงข้อมูล} คือคำสั่งกลุ่มใด? \\
    \textbf{ตอบ:} \rule{10cm}{0.4pt}
    
    \item จงระบุความสัมพันธ์ (Cardinality) ของหัวข้อดังต่อไปนี้ว่าเป็น one-to-one / one-to-many / many-to-one / many-to-many:
    \begin{itemize}
        \item Mother - Child: \rule{6cm}{0.4pt}
        \item Doctor - Patient: \rule{6cm}{0.4pt}
        \item House - Address: \rule{6cm}{0.4pt}
    \end{itemize}
    
    \item จากข้อข้างบน (ความสัมพันธ์ 3 ข้อ) จะระบุว่ามีความสัมพันธ์ใดบ้างที่เป็น \textbf{Weak relationship}? \\
    \textbf{ตอบ:} \rule{10cm}{0.4pt}
    
    \item จงอธิบายสั้นๆ ว่า "Enterprise Integrity Constraint" คืออะไร? \\
    \textbf{ตอบ:} \rule{10cm}{0.4pt}
    
    \item Composite Key หมายความว่าอย่างไร? \\
    \textbf{ตอบ:} \rule{10cm}{0.4pt}
    
    \item หากต้องการลบข้อมูลเฉพาะเจาะจงในคำสั่ง DELETE ต้องเติม Clause ใด? \\
    \textbf{ตอบ:} \rule{10cm}{0.4pt}
    
    \item ใน EERD วงกลมที่มีอักษร "o" ย่อมาจากอะไรและหมายความว่าอย่างไร? \\
    \textbf{ตอบ:} \rule{10cm}{0.4pt}
    
    \item "Derived Attribute" แตกต่างจาก "Stored Attribute" อย่างไร? \\
    \textbf{ตอบ:} \rule{10cm}{0.4pt}
\end{itemize}

\newpage

\section*{Part 3: 2 long questions (ภาษาไทย)}
\textbf{คำสั่ง:} จงแสดงวิธีทำและคำตอบให้ถูกต้อง (ข้อละ 10 คะแนน)

\begin{enumerate}
    \item จงเขียนคำสั่ง \textbf{ทั้งหมด} ในการสร้างตารางพร้อมเพิ่มตัวอย่างข้อมูลทั้งสามแถวให้เหมือนภาพข้างล่าง โดยกำหนดให้
    \begin{enumerate}[label=\alph*.]
        \item \texttt{student\_id} เป็น primary key
        \item \texttt{firstname} และ \texttt{lastname} ต้องไม่เป็น NULL
    \end{enumerate}

    \begin{table}[H]
    \centering
    \begin{tabular}{|l|l|l|l|l|}
    \hline
    \textbf{student\_id} & \textbf{firstname} & \textbf{lastname} & \textbf{age} & \textbf{tel} \\ \hline
    27001 & Luffy & Monkey & 18 & 1234567890 \\ \hline
    27002 & Chopper & Tony & NULL & 0123456789 \\ \hline
    27003 & Roronoa & Zoro & 17 & NULL \\ \hline
    \end{tabular}
    \end{table}

    \textbf{พื้นที่เขียนคำตอบ:}
    \begin{lstlisting}[frame=single, minheight=8cm]
-- Write SQL here

    \end{lstlisting}

    \newpage
    \item บริษัทแห่งหนึ่งต้องการสร้างฐานข้อมูล โดยกำหนดให้สร้าง \textbf{Conceptual level design} โดยใช้ \textbf{Chen notation ER diagram} จากข้อมูลต่อไปนี้
    \begin{enumerate}[label=\alph*.]
        \item บริษัทมีหลายแผนก (\texttt{Department}) ซึ่งเก็บรหัสและชื่อแผนกโดยชื่อไม่ซ้ำกัน
        \item แต่ละแผนกมีพนักงาน (\texttt{Employee}) โดยเก็บข้อมูลรหัสพนักงาน ชื่อ นามสกุล ที่อยู่ อายุ ซึ่งพนักงานแต่ละคนสังกัดได้เพียงแผนกเดียวเท่านั้น (1:N)
        \item ในกรณีที่พนักงานมีลูก (\texttt{Children}) ให้เก็บชื่อลูกและวันเกิด (ถือเป็น Weak Entity)
    \end{enumerate}
    
    \vspace{0.5cm}
    \centerline{\fbox{\begin{minipage}[c][12cm]{15cm} \centering (วาด ER Diagram ตรงนี้ - เน้นความถูกต้องของสัญลักษณ์ Weak Entity) \end{minipage}}}
\end{enumerate}

\newpage
\begin{center}
    \rule{\textwidth}{2pt} \\
    \textbf{\Large เฉลยละเอียด (Senior Practice Solution Key)} \\
    \rule{\textwidth}{2pt}
\end{center}

\subsection*{Part 1: MCQ Explanations}
\begin{longtable}{|c|c|p{10.5cm}|}
\hline \textbf{No.} & \textbf{Ans} & \textbf{Rational / Explanation (ภาษาไทย)} \\ \hline
1 & c & Relation ห้ามมีข้อมูลซ้ำกันทั้งแถว (Duplicated rows) เพื่อให้ระบุตัวตนได้ \\ \hline
2 & b & E.F. Codd จาก IBM เป็นผู้เสนอ Relational Model \\ \hline
3 & b & Composite attribute คือแอตทริบิวต์ที่แตกย่อยออกไปได้อีก \\ \hline
4 & c & M:N Mapping ต้องมี 3 ตาราง (2 ตาราง Entity และ 1 ตาราง Junction) \\ \hline
5 & a & 1:1 มีความจุที่สามารถรวมเป็นตารางเดียวได้เพื่อลดความซับซ้อน \\ \hline
6 & a & การยุบรวมตาราง (Mapping) มักทำกับ 1:1 เพื่อประสิทธิภาพ \\ \hline
7 & d & WHERE ใช้กำหนดเงื่อนไขแถวที่ต้องการลบ \\ \hline
8 & b & Double Rectangle = Weak Entity ใน Chen Notation \\ \hline
9 & c & ALTER TABLE + ADD คือการเพิ่มคอลัมน์ \\ \hline
10 & c & Internal Schema จัดการการเก็บไฟล์จริงๆ (Indexing/Hashing) \\ \hline
11 & b & Atomic attribute คือค่าเดี่ยวที่ไม่สามารถแยกย่อยได้อีก \\ \hline
12 & c & Double Line ใน ER หมายถึงทุกตัวตนต้องเข้าร่วม (Total Participation) \\ \hline
13 & c & Partial Key (Dashed Line) ใช้บอกความต่างของ Weak Entity ในกลุ่ม Owner เดิม \\ \hline
14 & b & DDL (Data Definition Language) ใช้จัดการโครงสร้างฐานข้อมูล \\ \hline
15 & b & Foreign Key วางที่ฝั่ง Many เพื่ออ้างอิงกลับไปยังฝั่ง One \\ \hline
16 & c & Primary Key ต้องไม่ซ้ำ (Unique) และไม่ว่าง (Not Null) ตามกฎ Entity Integrity \\ \hline
17 & b & DELETE ลบข้อมูลในตาราง (TRUNCATE เร็วกว่าแต่ในตัวเลือกใช้ DELETE) \\ \hline
18 & c & Recursive Relationship คือความสัมพันธ์ที่ Entity กระทำกับตนเอง \\ \hline
19 & b & Referential Integrity บังคับความถูกต้องของคีย์นอก (Foreign Key) \\ \hline
20 & b & COMMIT ใช้ยืนยันการเปลี่ยนแปลงใน Transaction \\ \hline
\end{longtable}

\subsection*{Part 2: Short Answer Key}
\begin{itemize}
    \item \texttt{ALTER TABLE}
    \item **MySQL** หรือ **MariaDB**
    \item **DML** (Data Manipulation Language)
    \item Mother-Child = **one-to-many**, Doctor-Patient = **many-to-many**, House-Address = **one-to-one**
    \item **Mother-Child** (Child ในเชิงธุรกิจมักเป็น Weak Entity ที่ขึ้นกับพนักงาน/แม่)
    \item กฎเกณฑ์ที่ธุรกิจกำหนดเองเพื่อคุมความถูกต้องข้อมูล (เช่น อายุต้อง > 18)
    \item Primary Key ที่ประกอบด้วยคุณสมบัติหลายตัวรวมกัน
    \item \texttt{WHERE}
    \item \textbf{Overlapping} - Entity หนึ่งสามารถอยู่ใน Subclass หลายกลุ่มได้พร้อมกัน
    \item Derived คือค่าที่คำนวณได้ (ไม่เก็บจริง) ส่วน Stored คือค่าที่บันทึกลงดิสก์
\end{itemize}

\subsection*{Part 3: Long Answer Solution}
\textbf{1. SQL Solution:}
\begin{lstlisting}[language=SQL]
CREATE TABLE students (
    student_id INT PRIMARY KEY,
    firstname VARCHAR(50) NOT NULL,
    lastname VARCHAR(50) NOT NULL,
    age INT,
    tel VARCHAR(15)
);

INSERT INTO students VALUES (27001, 'Luffy', 'Monkey', 18, '1234567890');
INSERT INTO students VALUES (27002, 'Chopper', 'Tony', NULL, '0123456789');
INSERT INTO students VALUES (27003, 'Roronoa', 'Zoro', 17, NULL);
\end{lstlisting}

\textbf{2. ER Diagram Concept:}
\begin{center}
\begin{tikzpicture}[node distance=2.5cm, every edge/.style={draw, thick}]
    \node[entity] (dept) {Department};
    \node[relationship] (works) [right=of dept] {Works\_In};
    \node[entity] (emp) [right=of works] {Employee};
    \node[identRelationship] (has) [below=of emp] {Has};
    \node[weakEntity] (child) [below=of has] {Children};
    
    \draw (dept) -- (works) node[midway, above] {1};
    \draw (works) -- (emp) node[midway, above] {N};
    \draw (emp) -- (has) node[midway, left] {1};
    \draw[double, thick] (has) -- (child) node[midway, left] {N};
    
    \node[keyAttribute] (did) [above=0.5cm of dept] {\underline{ID}};
    \node[attribute] (dname) [below=0.5cm of dept] {Name};
    \node[keyAttribute] (eid) [above=0.5cm of emp] {\underline{EmpID}};
    \node[partialKey] (cname) [right=0.5cm of child] {\dashuline{Name}};
    
    \draw (dept) -- (did); \draw (dept) -- (dname);
    \draw (emp) -- (eid); \draw (child) -- (cname);
\end{tikzpicture}
\end{center}

\end{document}
