\documentclass[a4paper,10pt]{article}

% Packages
\usepackage{fontspec}
\usepackage{xunicode}
\usepackage{xltxtra}
\usepackage[margin=2cm]{geometry}
\usepackage{enumitem}
\usepackage{listings}
\usepackage[table]{xcolor}
\usepackage{graphicx}
\usepackage{longtable}
\usepackage{tikz}
\usetikzlibrary{shapes.geometric, arrows, positioning, fit, calc, backgrounds}

% Fonts
\setmainfont{TH Sarabun New}
\setmonofont{Courier New}

% Code Style
\definecolor{codegreen}{rgb}{0,0.6,0}
\definecolor{codegray}{rgb}{0.5,0.5,0.5}
\definecolor{codepurple}{rgb}{0.58,0,0.82}
\definecolor{backcolour}{rgb}{0.95,0.95,0.92}

\lstdefinestyle{mystyle}{
    backgroundcolor=\color{backcolour},   
    commentstyle=\color{codegreen},
    keywordstyle=\color{blue},
    numberstyle=\tiny\color{codegray},
    stringstyle=\color{codepurple},
    basicstyle=\ttfamily\footnotesize,
    breakatwhitespace=false,         
    breaklines=true,                 
    captionpos=b,                    
    keepspaces=true,                 
    numbers=left,                    
    numbersep=5pt,                  
    showspaces=false,                
    showstringspaces=false,
    showtabs=false,                  
    tabsize=2
}
\lstset{style=mystyle, language=SQL}

% TikZ Styles for ERD
\tikzstyle{entity} = [rectangle, draw, text centered, minimum height=2em, minimum width=3em]
\tikzstyle{weakentity} = [rectangle, draw, double, double distance=2pt, text centered, minimum height=2em, minimum width=3em]
\tikzstyle{relationship} = [diamond, draw, text centered, aspect=2]
\tikzstyle{identrelationship} = [diamond, draw, double, double distance=2pt, text centered, aspect=2]
\tikzstyle{attribute} = [ellipse, draw, text centered, node distance=2cm]
\tikzstyle{keyattribute} = [ellipse, draw, text centered, node distance=2cm, font=\bfseries]
\tikzstyle{multiattribute} = [ellipse, draw, double, double distance=2pt, text centered]
\tikzstyle{dashattribute} = [ellipse, draw, dashed, text centered]

% Dash underline command
\usepackage[normalem]{ulem}
\def\dashuline{\bgroup
  \ifdim\ULdepth=\maxdimen  % Set depth if not set
   \settodepth\ULdepth{(j}\advance\ULdepth.4pt\fi
  \markoverwith{\kern.15em\vtop{\kern\ULdepth \hrule width .2em}\kern.15em}\ULon}

\title{CPE241 Module 1: Ultimate Cheat Sheet (Mock Exam Reference)}
\author{Generated for Exam Prep}
\date{February 19, 2026}

\begin{document}

\maketitle

\section*{Part 1: Core Concepts (Expanded \& Related)}

\subsection*{1. Database Pioneers \& Models}
\begin{itemize}
    \item \textbf{Dr. Edgar F. Codd (E.F. Codd):} The "Father" of the Relational Model (1970). Proposed the 12 rules for RDBMS.
    \item \textbf{Peter Chen:} The creator of the Entity-Relationship (ER) Model (1976).
    \item \textbf{Charles Bachman:} Developer of the Network Model (IDS). Known for Bachman diagrams.
\end{itemize}

\subsection*{2. Characteristics of Relations (Tables)}
\begin{itemize}
    \item \textbf{Rows (Tuples):}
    \begin{itemize}
        \item \textbf{Uniqueness:} No duplicate rows allowed (Key concept).
        \item \textbf{Order:} Row order is \textbf{insignificant}.
    \end{itemize}
    \item \textbf{Columns (Attributes):}
    \begin{itemize}
        \item \textbf{Atomic:} Cells must have single values (1NF).
        \item \textbf{Homogenous:} All values in a column must be same type.
        \item \textbf{Order:} Column order is \textbf{insignificant}.
    \end{itemize}
\end{itemize}

\subsection*{3. Attribute Classifications \& Notation}
\begin{itemize}
    \item \textbf{Simple:} Atomic, cannot be split (e.g., Blood Type). \textit{Solid Ellipse}.
    \item \textbf{Composite:} Can be split into sub-attributes (e.g., Name $\rightarrow$ First, Last).
    \item \textbf{Multi-valued:} Has multiple values for one entity. \textit{Double Ellipse}. (Requires new table).
    \item \textbf{Derived:} Calculated from others (e.g., Age from Birthdate). \textit{Dashed Ellipse}.
    \item \textbf{Stored:} Attributes physically saved in the DB.
\end{itemize}

\subsection*{4. Keys \& Identification}
\begin{itemize}
    \item \textbf{Super Key:} Any set of attributes that uniquely identifies a row.
    \item \textbf{Candidate Key:} A \textbf{minimal} Super Key.
    \item \textbf{Primary Key (PK):} The chosen candidate key. Not NULL.
    \item \textbf{Foreign Key (FK):} Links to a PK in another table.
\end{itemize}

\subsection*{5. ER Notation \& Weak Entities (Chen Model)}
\begin{itemize}
    \item \textbf{Strong Entity:} Rectangle. Independent existence.
    \item \textbf{Weak Entity:} Double Rectangle. Dependent existence.
    \begin{itemize}
        \item \textit{Identification:} Identified by its \textbf{Partial Key} + Owner's \textbf{PK}.
    \end{itemize}
    \item \textbf{Identifying Relationship:} Double Diamond. Connects Weak to Owner.
    \item \textbf{Total Participation:} Double Line. Entity \textit{must} participate (min=1).
\end{itemize}

\subsection*{6. Mapping Relationships}
\begin{itemize}
    \item \textbf{1:1:} Collapse into 1 table if Total Participation on both sides. Else add FK to Total side.
    \item \textbf{1:N:} FK goes to \textbf{"Many"} side.
    \item \textbf{M:N:} Create a \textbf{New Junction Table}. Composite PK (FK\_A + FK\_B).
\end{itemize}

\subsection*{7. Integrity Constraints}
\begin{itemize}
    \item \textbf{Entity Integrity:} PK cannot be NULL + Must be Unique.
    \item \textbf{Referential Integrity:} FK must match Parent PK or be NULL.
    \item \textbf{Domain Constraint:} Values must match defined type.
\end{itemize}

\subsection*{8. EER Constraints}
\begin{itemize}
    \item \textbf{Disjoint (d):} Entity must be \textbf{only one} subclass.
    \item \textbf{Overlapping (o):} Entity can be \textbf{multiple} subclasses.
\end{itemize}

\subsection*{9. SQL Categories}
\begin{itemize}
    \item \textbf{DDL:} \texttt{CREATE}, \texttt{ALTER}, \texttt{DROP}, \texttt{TRUNCATE}.
    \item \textbf{DML:} \texttt{INSERT}, \texttt{UPDATE}, \texttt{DELETE}, \texttt{SELECT}.
    \item \textbf{DCL:} \texttt{GRANT}, \texttt{REVOKE}.
\end{itemize}

\subsection*{10. Modification Rules}
\begin{itemize}
    \item \textbf{DELETE requires WHERE:} Without \texttt{WHERE}, it deletes all rows.
    \item \textbf{UPDATE requires WHERE:} Without \texttt{WHERE}, it updates all rows.
\end{itemize}

\hrule

\section*{Part 2: Short Answer Concepts}

\begin{itemize}
    \item \textbf{Schema Modification (ALTER):}
    \begin{itemize}
        \item \texttt{ALTER TABLE t ADD column type;}
        \item \texttt{ALTER TABLE t MODIFY column type;}
        \item \texttt{ALTER TABLE t DROP COLUMN column;}
        \item \texttt{ALTER TABLE t RENAME TO new\_name;}
    \end{itemize}
    \item \textbf{Deletion Commands:}
    \begin{itemize}
        \item \textbf{DELETE:} SQL DML. Slow.
        \item \textbf{TRUNCATE:} SQL DDL. Fast (Resets ID).
        \item \textbf{DROP:} SQL DDL. Removes Table completely.
    \end{itemize}
    \item \textbf{Weak Relationship:} A relationship where the child cannot exist without the parent (e.g., Hotel - Room).
    \item \textbf{Composite Key:} PK made of 2+ attributes. Essential for M:N junction tables.
\end{itemize}

\hrule

\section*{Part 3: Simulation Practice (Long Answer Style)}

\subsection*{Practice Problem 1: SQL Design (Table Recreation)}
\textbf{Problem:} Write SQL to create table and insert 2 rows.

\begin{center}
\begin{tabular}{|c|c|c|c|}
\hline
\textbf{member\_id} & \textbf{username} & \textbf{email} & \textbf{points} \\ \hline
1001 & gamer\_x & x@game.com & 500 \\ \hline
1002 & pro\_player & pro@game.com & NULL \\ \hline
\end{tabular}
\end{center}

\textbf{Answer:}
\begin{lstlisting}
CREATE TABLE Members (
    member_id INT PRIMARY KEY,
    username VARCHAR(50) NOT NULL,
    email VARCHAR(100) NOT NULL,
    points INT
);

INSERT INTO Members VALUES (1001, 'gamer_x', 'x@game.com', 500);
INSERT INTO Members VALUES (1002, 'pro_player', 'pro@game.com', NULL);
\end{lstlisting}

\subsection*{Practice Problem 2: ER Conceptual Design}
\textbf{Problem Scenario:} Design an ER Diagram for \textbf{Company Project System}.
\begin{itemize}
    \item A \textbf{Department} controls many \textbf{Projects} (1:N).
    \item Each Project has multiple \textbf{Tasks} (Weak Entity).
\end{itemize}

\textbf{Diagram Solution:}

\begin{center}
\begin{tikzpicture}[node distance=2.5cm, every edge/.style={draw, thick}, scale=0.9, transform shape]
    % Entities
    \node[entity] (dept) {Department};
    \node[entity] (proj) [right=4cm of dept] {Project};
    \node[weakentity] (task) [below=3cm of proj] {Task};

    % Attributes for Dept
    \node[keyattribute] (dname) [above left=0.5cm and 0.5cm of dept] {\underline{Name}} edge (dept);
    \node[attribute] (dloc) [below left=0.5cm and 0.5cm of dept] {Location} edge (dept);

    % Attributes for Project
    \node[keyattribute] (ptitle) [above=0.8cm of proj] {\underline{Title}} edge (proj);
    \node[attribute] (pbudget) [right=0.8cm of proj] {Budget} edge (proj);

    % Attributes for Task (Weak)
    \node[attribute] (taskno) [left=1.5cm of task] {\dashuline{TaskNo}} edge (task); % Partial Key
    \node[attribute] (tdesc) [right=1.5cm of task] {Desc} edge (task);

    % Relationships
    \node[relationship] (controls) at ($(dept)!0.5!(proj)$) {Controls};
    \node[identrelationship] (has) at ($(proj)!0.5!(task)$) {Has};

    % Connectors
    \draw (dept) -- (controls) node[pos=0.3, above=0.1cm] {1};
    \draw (controls) -- (proj) node[pos=0.7, above=0.1cm] {N};

    \draw (proj) -- (has) node[pos=0.3, left=0.1cm] {1};
    \draw[double, thick] (has) -- (task) node[pos=0.7, left=0.1cm] {N}; % Total Participation

\end{tikzpicture}
\end{center}

\end{document}
